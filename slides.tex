\documentclass{beamer}

\usepackage{ulem}

% Information to be included in the title page:
\title{Embedding Lambda Calculus into Haskell with Safety Features}
\author{Seungheon Oh, Tomasz Maciosowski}
\institute{MLabs}
\date{Jan 2025}

\begin{document}

\frame{\titlepage}

\begin{frame}
  \frametitle{Background}
  Distributed computing system that executes modified version of $\lambda$-calculus is provided for our research domain specific language (DSL) to target. \\~\\

  $\lambda$-caluclus allows
  \begin{enumerate}
  \item
    Simple reduction laws that allows high reproducibility
    % Ease of evaluation on each node of distributed computing
  \item
    Easy to encoding and expansion feature sets
  \item
    \sout{Easy}Easier to target as compilation result
  \end{enumerate}

  %
\end{frame}

\begin{frame}[fragile]
  \frametitle{Example: Pseudo code}
  (($\lambda .$(($\lambda .$(($\lambda .$(2($\lambda .$((2 2) 1))))($\lambda .$(2($\lambda .$((2 2) 1))))))
  ($\lambda .$($\lambda .$(force(((3(force((($\lambda .$($\lambda .$(((5 2) 1)(delay(con bool False)))))((equals 1)(integer 1)))(delay((equals 1)(integer 2))))))(delay(integer 1)))(delay(((builtin addInteger)(2(1 - (integer 1))))(2(1 - (integer 2))))))))))) ifThenElse))
\end{frame}

\begin{frame}
  \frametitle{Embedding into Haskell}
  \begin{enumerate}
  \item
    Each ``terms'' should have explicit types
  \item
    Terms with and without unbound variables needs to be distinguished at the type level
  \end{enumerate}
\end{frame}

\begin{frame}
  \frametitle{Embedding into Haskell - unbound variables}

\begin{verbatim}

\begin{frame}
  \frametitle{Embedding into Haskell - unbound variables}

\begin{verbatim}

\end{verbatim}
\end{frame}

\begin{frame}
  \frametitle{Considerations for the VM}
  \begin{enumerate}
  \item
    Evaluation is eager(Call by value)
  \item
    C
  \end{enumerate}
\end{frame}

\begin{frame}
  \frametitle{Optimizations}
  \begin{itemize}
    \item
    \item Optimizing for both size and execution units
    \item Execution VM has strict limits on both
    \end{itemize}
\end{frame}

\begin{frame}
\frametitle{Optimizing Funciton Application}
\begin{verbatim}
papp :: Term s (a :--> b) -> Term s a -> Term s b
papp x y = Term \i ->
  (,) <$> asRawTerm x i <*> asRawTerm y i >>= \case
    -- Applying anything to an error is an error.
    (getTerm -> RError, _) -> pure $ mkTermRes RError
    -- Applying an error to anything is an error.
    (_, getTerm -> RError) -> pure $ mkTermRes RError
    -- Applying to `id` changes nothing.
    (getTerm -> RLamAbs 0 (RVar 0), y') -> pure y'
    (getTerm -> RHoisted (HoistedTerm _ (RLamAbs 0 (RVar 0))), y') -> pure y'
    -- new RApply
    (x', y') -> pure $ TermResult (RApply (getTerm x') [getTerm y']) (getDeps x' <> getDeps y')
\end{verbatim}
\end{frame}

\begin{frame}
  \frametitle{\verb|let| Bindings}
  \begin{itemize}
  \item UPLC does not have native let bindings
  \item Can be simulated with
\begin{verbatim}
let x = f y in g x x

(\x -> g x x) (f y)
\end{verbatim}
\end{frame}

\begin{frame}
  \frametitle{Optimizing \verb|let| Bindings}
  \begin{itemize}
\item Let binding can be eliminated if term is small enough
    \end{itemize}
\begin{verbatim}
plet :: Term s a -> (Term s a -> Term s b) -> Term s b
plet v f = Term \i ->
  asRawTerm v i >>= \case
    (getTerm -> RVar _) -> asRawTerm (f v) i
    (getTerm -> RBuiltin _) -> asRawTerm (f v) i
    (getTerm -> RHoisted _) -> asRawTerm (f v) i
    _ -> asRawTerm (papp (plam' f) v) i
\end{verbatim}
\end{frame}

\begin{frame}
  \framtitle{Hoisting}
  \begin{itemize}
    \item By default every term is inlined at every use
    \item Closed terms can be hoisted - transformed into top-level let-binding
    \item During plutarch compilation terms are hashed and compared against hoisted terms
    \item If hoisted term is small enough or used only once it will get inlined anyway
    \end{itemize}
\begin{verbatim}
phoistAcyclic :: ClosedTerm a -> Term s a
\end{verbatim}
\end{frame}

\end{document}
